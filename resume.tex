% !TEX program = xelatex

\documentclass{chicv}

% optionally suppress printing the page number
\pagenumbering{gobble}

\begin{document}

%% basic personal info
\name{Fan Nie}
\begin{basicinfo}
  \info{\email{niefan1208@gmail.com}}
  % \info{\homepage{skyzh.dev}[https://skyzh.dev]}
  \info{\github{fannie1208}[https://github.com/fannie1208]}
  % \info{\linkedinsquare{alex-chi-skyzh}[https://www.linkedin.com/in/alex-chi-skyzh/?originalSubdomain=cn]}
  \info{\phone{15317918806}}
\end{basicinfo}

\section{Education}
\cventry{Shanghai Jiao Tong University}
{Sep 2020 -- Jun 2024 (Expected)}
[B.Eng in Computer Science and Technology (IEEE Honor Class)]
[Shanghai, China]
\begin{itemize}
	\item GPA 93.38/100, Rank 2/127
	\item A+ Courses: Data Structure, Operating System, Computer Architecture, Computer Networks, and 19 others
\end{itemize}

\cventry{École Polytechnique Fédérale de Lausanne(EPFL)}
  {Feb 2023 -- Aug 2023}
  [Exchange Student of Computer Science]
  [Lausanne, Switzerland]
  \begin{itemize}
  	\item Courses: Database System(6.0/6.0), Machine Learning(6.0/6.0), Data Visualization(6.0/6.0)
  \end{itemize}

\section{Research Experience}

\cventry{Uncertainty-Aware Decision Transformer}
{Mar 2023 - Present}
[CoRL 2023 submission; the Co-First Author]
[MARSLab, THU]

\begin{itemize}
	\item Present an uncertainty-aware decision transformer for stochastic driving environment which can down-weight actions that result in highly uncertain returns to enable DT to focus on learning actions that can accurately achieve target returns.
	\item Paper; Code; Experiment; Visualization
	\item Extensive experimental results demonstrate UNREST's superior performance in various driving scenarios and the power of our uncertainty estimation strategy.
\end{itemize}

%\cventry{Skill-Based Offline Motion Planning}
%{Dec 2022 - Present}
%[will submit to NeurIPS 2023; the Second Author]
%[MARSLab, THU]

%\begin{itemize}
%	\item Present a skill-based framework that enhances offline RL to overcome the challenge of long-horizon planning.
%	\item Coding; Do experiment; Visualization of extracted skills and experimental results; Paper writing.
%	\item \textbf{Keywords: Auto Driving, RL, PyTorch}
%\end{itemize}

\cventry{Improving Generalization of GNNs with Divergence Fields Decomposition.}
{Oct 2022 - May 2023}
[NeurIPS 2023 Submission; the Second Author]
[Thinklab, SJTU]

\begin{itemize}
	\item Propose a graph diffusion model with branching-structured divergence fields to improve generalization of GNNS.
	\item Set up the pipeline and models; Do comparative and ablation experiments to show the model performance; Visualization of experimental results;
	\item Our GINN perform best in various node property prediction tasks, where training and testing distributions exhibit significant differences, with up to 27.4\% improvement over state-of-the-art models.
\end{itemize}


\cventry{Simplified Graph Transformers Inspired by Gradient Flows.}
{July 2022 - Jan 2023}
[ICML 2023 Submission; the Fourth Author]
[Thinklab, SJTU]

\begin{itemize}
	\item Propose simplified graph Transformers (SGFormer) as a powerful and scalable encoder for large graphs. SGFormer resorts to simple-yet-effective designs: one or two-layer feature propagation and global attention with linear complexity.
	\item Set up baselines; Do comparative and ablation experiments to show the model performance; Visualization of experimental results;
	\item Our SGFormer significantly outperforms SOTA Transformers, with 74x acceleration in terms of training time costs
\end{itemize}

\cventry{GraphDE: A Generative Framework for Debiased Learning and Out-of-Distribution Detection on Graph Data.}
{Mar 2022 - Sep 2022}
[full paper accepted by NeurIPS 2022; the Third Author]
[Thinklab, SJTU]

\begin{itemize}
	\item Tackle the problems of outliers in training set and OOD samples from new data in graph data under a unified probabilistic model; Automatically identify and down-weight outliers in the training stage and induce a OOD detector from the model.
	\item Set up baselines; Conduct 15+ experiments on different datasets to show the model performance and robustness; Visualization of experimental results; Draft and finalize the paper.
	\item Our model GraphDE achieves consistent performance improvements over the baselines. For example, in the OOD detection task, it outperforms the strongest baseline by 9.31\% on MNIST-75sp.
\end{itemize}

\section{Internship Experience}
\cventry{Qizhi Institute.}
{July 2023 – Present}
[Research Intern]
[Shanghai, China]

\begin{itemize}
	\item Supervised by Prof. Hang Zhao.
	\item Research on multimodal learning and autonomous driving.
\end{itemize}

\cventry{Biomap, Inc.}
{Jun 2022 – Dec 2022}
[Algorithm R\&D Intern]
[Beijing, China]

\begin{itemize}
	\item Set up the DeepCellState baseline and different types of Attention Free models  to predict changes in gene expression levels after drug interference using PyTorch, and test their performance on our datasets.
	\item Design and implement data binning, resulting in smaller losses.
	\item Finetune the pretrained model and raise the f scores.
\end{itemize}

\section{Project Experience}
\cventry{Coffee Kingdom Visualization.}
  {Apr 2023 - June 2023}
  [Course Project for Data Visualization]
  [\iconlink[\faGithub][fannie1208/project-2023-kingdom\_of\_kaffa]{https://github.com/fannie1208/project-2023-kingdom\_of\_kaffa}]
\begin{itemize}
	\item Design and implement an Interactive visualization website to assist those coffee lovers in finding the perfect package of freshly roasted coffee.
\end{itemize}

\cventry{Graph Neural Networks for Scalable Combinatorial Optimization.}
{Mar 2023 - June 2023}
[Research Project in LIONS, EPFL]
[]

\begin{itemize}
	\item Speed up the decoding process of solving CO problems with a GNN by directly sampling from the learned probabilities and employ a STE to guide the network in making accurate discrete decisions.
	\item Code; Experiment; Paper Writing
\end{itemize}

\cventry{PLI-Python-based-Lambda-Interpreter}
  {Dec 2022 - Jan 2023}
  [Course Project for Programming Language]
  [\iconlink[\faGithub][FKCSP/PLI-Python-based-Lambda-Interpreter]{https://github.com/FKCSP/PLI-Python-based-Lambda-Interpreter}]
  \begin{itemize}
    \item Design and implement a lambda interpreter based on Python that supports arithmetic operations, size comparisons, conditional branches, and recursive functions.
  \end{itemize}

\section{Skills}

\begin{compactlist}
  \item \textbf{Programming Languages}: Python, C++, JavaScript, HTML, CSS
  \item \textbf{Tech Skills}: MySQL, PyTorch, Data Visualization, Web Development, Web Crawler
\end{compactlist}

\section{Extracurricular Activities}
\cventry{Youth Volunteer Team}
{Mar 2021 - Dec 2022}
[Minister of Planning]
[SJTU, Shanghai]

\begin{itemize}
	\item Plan and organize various volunteer activities such as Shanghai Marathon volunteers, etc.; Write planning cases and liaise with different departments.
\end{itemize}
\end{document}
